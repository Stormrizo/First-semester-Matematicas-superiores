\documentclass[12pt]{article} 
\usepackage[utf8]{inputenc}
\usepackage[spanish]{babel}
\usepackage{amsfonts}
\usepackage{amsmath, amsthm, amssymb}
\usepackage{graphicx}
\usepackage[left=2.54cm,right=2.54cm,top=2.54cm,bottom=2.54cm]{geometry}
\usepackage{pstricks}
\begin{document}

\thispagestyle{empty} 
\begin{center} \LARGE{\bf Benemérita Universidad Autónoma de Puebla} \\[0.5cm]
\begin{figure}[htb] \centering \includegraphics[scale=.2]{LogoBUAPpng.png} \end{figure}
\LARGE{Facultad de Ciencias Físico Matemáticas}\\[0.5cm]
\begin{figure}[htb] \centering \includegraphics[scale=.39]{LogoFCFMBUAP.png} \end{figure} 
\Large{Licenciatura en Física Teórica}\\[0.5cm]
\large{Primer semestre} \end{center}
\begin{center} { \Large \bfseries{Tarea 4: Resolver y entregar en un archivo PDF los ejercicios del libro de cálculo del Lang:}} \\ \end{center}
\large{\bf Curso: } Matemáticas Superiores \textbf{(N.R.C.:25601)}\\
\large{\bf Alumno:} Julio Alfredo Ballinas García $\left(202107583\right)$ \\
\large{\bf Docente:} Dr. Jorge Cotzomi Paleta\\
\large{\bf Grupo:} 102\\ \begin{center} 
\vfill
\textsc{12 de septiembre de 2021} \end{center}  
\newpage



\begin{center}
\Large{Resolver y entregar en un archivo PDF los ejercicios del libro de cálculo del Lang:} 
\end{center}
\sffamily
El siguiente trabajo trata sobre la solución de los siguientes problemas del libro \textbf{Cálculo} de  
\textit{Serge Lang} el cual se usa en la materia de Matemáticas Superiores, impartida por el Doctor Jorge Cotzomi Paleta.

\begin{figure}[htb] \centering \includegraphics[scale=.6]{Tarea4.png}
\caption{Lista de ejercicios por resolver}   
\end{figure}

\sffamily
Es necesario aclarar que los ejercicios 3, 4 y 5 sus páginas se repiten en la imagen, es por ello que un compañero de clase (Brandon Yahir Arriaga Tlapa) preguntó en el grupo de Whatsapp sobre esa cuestión, el profesor mencionó que debido a esa situación los enunciadps de los ejercicios debería de ser de la siguiente manera:

\begin{enumerate}
    \item [3.] Capítulo 2 sección 3. Página 31.
    \item [4.] Capítulo 2 sección 4. Página 33.
    \item [5.] Capítulo 2 sección 6. Página 37.
\end{enumerate}

\sffamily
A continuación de presentan las soluciones a estos problemas:

\newpage

\section{Capítulo 1 Sección 4, página 18: ejercicio 10}

\sffamily
Hallar $a^x$ y $x^a$ para los siguientes valores de $x$ y $a$.

\begin{itemize}
    \item $a = -\frac{1}{2}$
    \item $x = 9$
\end{itemize}

\begin{enumerate}
    \item [I.] $a^x$
\end{enumerate}
{\red Solución.} 

\begin{equation}
    \begin{split}
        a^x & = {(- \frac{1}{2} )}^9\\\\
         & = {(-\frac{1}{2})}^8 {(-\frac{1}{2})}^1 \\\\
        & = (\frac{(-1)^8}{2^8}) (\frac{(-1)^1}{2^1}) \\\\
         & = (\frac{(1)}{256}) (-\frac{(1)}{2}) \\\\
          & = (-\frac{1}{512})\\\\
    \end{split}
\end{equation}
La solución para $a^x$ es $a^x = (-\frac{1}{512}) $ \\
\newpage

\begin{enumerate}
    \item [II.] $x^a$ 
\end{enumerate}
{\red Solución.} 
\begin{equation}
    \begin{split}
        x^a & = 9^{(-\frac{1}{2})}\\\\
         & = (\frac{1}{9^{(\frac{1}{2})}})\\\\
        & = (\frac{1}{\sqrt{9}})\\\\
         & = (\frac{1}{|3|})\\\\
          & = (\frac{1}{3})\\\\
    \end{split}
\end{equation}
La solución para $x^a$ es \textbf{$x^a = (\frac{1}{3}) $} \\

\newpage

\section{Capítulo 2 Sección 2, página 26: ejercicio 34}

\sffamily
Esbozar la gráfica de las siguientes funciones:

\begin{enumerate}
      \item [I.] $f(x) = |x| + x$ si $-1\leq x\leq1$
    \item [II.] $f(x) = 3$ si $x>1$
\end{enumerate}

{\red Solución.} 

\begin{enumerate}
    \item [I.] $f(x) = |x| + x$ si $-1\leq x\leq1$
\end{enumerate}

Esta función $f(x) = |x| + x$ tomará un rango de valores en los reales $(\mathbb{R})$ de acuerdo con la condición: \\

\begin{center}
    $-1\leq x\leq1$
\end{center}

Entonces podemos decir que el $D_f = x \in \left[-1,1\right]$\\

El rango de esta función será $R_f = y \in \left[0,2\right]$ \\

\textbf{Gráfica:}

\begin{figure}[htb] \centering \includegraphics[scale=.55]{Grafica primera.png} \end{figure}
\newpage

\begin{enumerate}
    \item [II.] $f(x) = 3$ si $x>1$
\end{enumerate}

Esta función $f(x) = 3$  se trata de un caso especial de la ecuación de la recta, donde adopta la forma siguiente:
\begin{center}
$f(x)=k$, donde $k = 3$ en este caso.
\end{center}
\begin{center}
\textit{Tenemos la siguiente condición para esta} función:
\end{center}
\begin{center}
    $x>1$
\end{center}

Entonces podemos decir que el $D_f = x \in \left(1, \infty \right)$\\

El rango de esta función será $R_f = y \in \{3\}$ \\

\textbf{Gráfica:}

\begin{figure}[htb] \centering \includegraphics[scale=.55]{Gráfica 2.png} \end{figure}

\newpage

\section{Capítulo 2 Sección 3, página 31: ejercicio 25 y graficar}\\\\

¿Cuál es la ecuación de la recta que pasa por los puntos siguientes?

\begin{itemize}
    \item $\left(-1,2\right)$
    \item $\left(\sqrt{2},-1\right)$
\end{itemize}

{\red{Pre-solución.}} \\

Para poder empezar con la resolución de este problema es necesario dar una variable a cada coordenada. \\

Debemos tomar en cuenta que también debemos asignar a esas variables una posición respecto a la otra variable. Entonces sea A la cordenada con valores $\left(x_1,y_1\right)$ y B con $\left(x_2,y_2\right)$ .\\

Arbitrariamente podemos decir que tanto la coordenada $\left(-1,2\right)$ como la coordenada $\left(\sqrt{2},-1\right)$ pueden ser A o B. \\

Para este problema se propone que:

\begin{center}
    $A = \left(\sqrt{2},-1\right)$
\end{center}
\begin{center}
 y   
\end{center}
\begin{center}
    $B = \left(-1,2\right)$
\end{center}

Por lo tanto
\begin{center}
 $A = \left(\sqrt{2},-1\right) \Rightarrow x_1 = \sqrt{2}$ y $ y_1 = -1$ 
\end{center}
\begin{center}
 y
\end{center}
\begin{center}
 $B = \left(-1,2\right) \Rightarrow x_2 = -1$ y $ y_2 = 2$ 
\end{center}

Ahora, el problema menciona que se debe calcular la ecuación de la recta dados los puntos $A$ y $B$. \\

Para este problema vamos a usar la siguiente ecuación, la cual tiene por nombre \textit{Ecuación de la recta que pasa por dos puntos}.
\newpage

\begin{center}
 $y-y_1=\frac{y_2-y_1}{x_2-x_1}\cdot (x-x_1)$ 
\end{center}

En donde:
\begin{center}
La pendiente es:
 $ m = \frac{y_2-y_1}{x_2-x_1}$ 
\end{center}

Lo siguiente es empezar con la solución de este problema sustituyendo las cordenadas $(x_1,y_1)$ y $(x_2,y_2)$ en:

\begin{center}
 $y-y_1=\frac{y_2-y_1}{x_2-x_1}\cdot (x-x_1)$ 
\end{center}

{\red{Solución:}} \\ \\

Páginas (9), (10) y (11)  \\ \\  \\ \\  \\ \\  \\ \\

\begin{figure}[htb] \centering \includegraphics[scale=5]{descarga.png} \end{figure}

\newpage
\begin{equation}
    \begin{split}
       y-y_1=\frac{y_2-y_1}{x_2-x_1}\cdot (x-x_1) & =  y-(-1)=\frac{2-(-1)}{(-1)-(\sqrt{2})}\cdot (x-(\sqrt{2})\\\\
         & = y+1=\frac{3}{-1-\sqrt{2}}\cdot (x-\sqrt{2})\\\\ \textbf{Despeje de (-)}
        & = y+1=\frac{-3}{1+\sqrt{2}}\cdot (x-\sqrt{2})\\\\ \textbf{Pasamos} (1+\sqrt{2}) \textbf{al otro lado}
         & = (1+\sqrt{2}) \cdot (y+1)= -3\cdot (x-\sqrt{2})\\\\\textbf{Propiedad distributiva}
          & = (1+\sqrt{2}) \cdot (y+1)= -3\cdot (x-\sqrt{2})\\\\
            & = (y+\sqrt{2}\cdot y) + (1+\sqrt{2})= -3x+3\sqrt{2}\\\\
              & = (y+\sqrt{2}\cdot y) + (1+\sqrt{2}) +3x-3\sqrt{2} = 0\\\\ \textbf{Ec. general Ax+By+C}
                & = 3x+(1+\sqrt{2})\cdot y + 1-2\sqrt{2} = 0\\\\
                  & = (1+\sqrt{2})\cdot y =-3x - 1+2\sqrt{2} \\\\
                   & y =\frac{-3x - 1+2\sqrt{2}}{(1+\sqrt{2})} \\\\ \textbf{Conjugado de} (1+\sqrt{2})
                  & = y =\frac{-3x - 1+2\sqrt{2}}{(1+\sqrt{2})}\cdot \frac{(1-\sqrt{2})}{(1-\sqrt{2})}\\\\
                   & y =\frac{-3x - 1+2\sqrt{2}}{(1-2)}\cdot {(1-\sqrt{2})}\\\\
    \end{split}
\end{equation}

\newpage

\begin{equation}
    \begin{split}
         & y =\frac{-3x - 1+2\sqrt{2}}{(1-2)}\cdot {(1-\sqrt{2})}\\\\ 
        & y = \frac{(-3+3\sqrt{2})x - 1+\sqrt{2}+2\sqrt{2} -4}{(-1)}\\\\ 
         & y = \frac{(-3+3\sqrt{2})x - 5+3\sqrt{2}}{(-1)}\\\\\textbf{Ec. pendiente-ordenada al origen} 
        = & y = {(3-3\sqrt{2})x + 5-3\sqrt{2}}\\\\
    \end{split}
    \end{equation}
Vemos que obtenemos este resultado después de haber sustituido los valores de $(\sqrt{2},-1)$ y $(-1,2)$. 
\begin{center}
 Tenemos que gráficar esta ecuación:  
\end{center}

\begin{center}
    $y = {(3-3\sqrt{2})x + 5-3\sqrt{2}}$
\end{center}
\begin{center}
 Tiene la siguiente forma: 
\end{center}
\begin{center}
 $y=mx+b$
\end{center}
\begin{center}
 Donde: 
\end{center}
\begin{center}
$m={(3-3\sqrt{2})}$
\end{center}
\begin{center}
y
\end{center}
\begin{center}
$b=5-3\sqrt{2}$
\end{center}
Su gráfica es la siguiente:
\newpage

\textbf{Gráfica:} 

\begin{figure}[htb] \centering \includegraphics[scale=.55]{grafica4.png} \end{figure}

Vemos que el punto donde corta la gráfica cuando $x=0$ es la ordenada al origen;es decir $b$. De forma exacta $b=5-3\sqrt{2}$, pero de forma decimal es un valor aproximado a $0.7574$.\\\\\\


\section{Capítulo 2 Sección 4, página 33: ejercicio 9}

¿Cuáles son las longitudes de los lados del rectángulo del ejercicio 8?\\

\textbf{Ejercicio 8.} \textit{Hallar las coordenadas de la cuarta esquina de un rectángulo cuyas otras tres esquinas son $(-2,-2)$, $(3,-2)$, $(3,5)$}.\\

{\red{Pre-solución:}} \\

Para la resolución de este problema debemos utilizar los conceptos de segmento de recta y distancia entre dos puntos. Para ello vamos a usar la definición de un paralelogramo, la cual no dice:

\begin{center}
    [...] cuadrilátero cuyos pares de lados opuestos son iguales y paralelos dos a dos.
\end{center}

También usaremos las coordenas del problema 8 para la solución de este por medio de la siguiente ecuación:

\begin{center}
    $\overline{P_1P_2}=\sqrt{(x_2-x_1)^{2}+(y_2-y_1)^2}$
\end{center}

Es conveniente dar variables a nuestras coordenadas, para este caso, vamos a guiarnos de la posición que tienen en plano cartesiano, es decir respecto a los cuadrantes (I, II, III, IV), entonces podemos decir que:

\begin{center}
    $P_1 = (-2,-2)$
\end{center}
\begin{center}
    $P_2 = (3,-2)$
\end{center}
\begin{center}
    $P_3 = (3,5)$
\end{center}
\begin{center}
    $P_4 = (x_4,y_4)$
\end{center}

Podemos ver que la coordenada $P_1$ está en el cuadrante III, la coordenada $P_2$ en cuadrante (IV), $P_3$ en cuadrante I y por consiguiente la coordenada $P_4$ estará en el cuadrante (II).\\ 

Ahora sabemos por geometría que: La \textbf{base} del rectángulo sería el segmento de recta $\overline{P_1P_2}$ y la \textbf{altura} sería el segmento $\overline{P_2P_3}$, pero en base a la definición de un paralelogramo tenemos las siguientes equivalencias entre los segmentos de delimitan la figura:
\begin{center}
     $\overline{P_1P_2}=\overline{P_3P_4}$
\end{center}
\begin{center}
     y
\end{center}
\begin{center}
     $\overline{P_2P_3}=\overline{P_1P_4}$
\end{center}
Sabiendo esto, ahora podemos empezar con la solución de nuestro problema.\\
\newpage

{\red{Solución:}}\\

Vamos a usar la ecuación de distancia entre dos puntos para conocer el valor de la \textbf{base} y la \textbf{altura} de nuestro rectángulo:

\begin{center}
     $\overline{P_1P_2}=\sqrt{(x_2-x_1)^{2}+(y_2-y_1)^2}$
\end{center}

(I) Cálculo de base - segmento $\overline{P_1P_2}$ \\

 $P_1 = (-2,-2)$\\
 
 $P_2 = (3,-2)$

\begin{equation}
    \begin{split}
       \overline{P_1P_2}=\sqrt{(x_2-x_1)^{2}+(y_2-y_1)^2} & =  \sqrt{(3-(-2))^{2}+(-2-(-2))^2} \\\\
         & = \sqrt{(3-(-2))^{2}+(-2+2))^2}\\\\ 
         & = \sqrt{(3+2)^{2}}\\\\ 
        & = \sqrt{(5)^{2}}\\\\ \textbf{La base es} 
         & = 5u \\\\
    \end{split}
\end{equation}

(I) Cálculo de altura - segmento $\overline{P_2P_3}$ \\

 $P_2 = (3,-2)$\\
 
 $P_3 = (3,5)$

\newpage

\begin{equation}
    \begin{split}
       \overline{P_2P_3}=\sqrt{(x_3-x_2)^{2}+(y_3-y_2)^2} & =  \sqrt{(3-(3))^{2}+(5-(-2))^2} \\\\
         & = \sqrt{(3-3)^{2}+(5+2)^2}\\\\ 
         & = \sqrt{(5+2)^{2}}\\\\ 
        & = \sqrt{(7)^{2}}\\\\ \textbf{La altura es} 
         & = 7u \\\\
    \end{split}
\end{equation}

Hemos concluido que las longitudes del rectángulo son:

\begin{center}
    \textbf{Base} $=5u$
\end{center}
\begin{center}
    y
\end{center}
\begin{center}
   \textbf{Altura} $=5u$
\end{center}

\section{Capítulo 2 Sección 6, página 37: ejercicio 4}

Esbozar la gráfica de las ecuaciones siguientes:

\begin{center}
    $x^{2}+y^{2}-2x+3y-10=0$
\end{center}

{\red{Pre-solución}}\\

Esta ecuación se trata de la ecuación de la circunferencia con centro fuera del origen $C(h,k)$ y radio $r$, esto lo podemos deducir porque en la ecuación anterior hay términos como $-2x$ y $3y$, esto quiere decir que la fórmula está representada en su forma general $Ax^2+Cy^2+Dx+Ey+F=0$. \\

El problema nos pide esbozar la gráfica de la ecuación, esto se traduce en que debemos encontrar las coordenadas del centro de la circunferencia (para este caso es $C(h,k)$) y el valor del radio.

{\red{Solución}}\\

Tenemos la siguiente expresión:

\begin{center}
     $x^{2}+y^{2}-2x+3y-10=0$
\end{center}

Utilicemos la \textbf{propiedad asociativa} para reunir a los terminos que tienen la variable $x$ y la variable $y$, los términos constantes los despejamos.\\

Nos queda:

\begin{center}
     $x^{2}+-2x+y^{2}+3y=10$
\end{center}

Completamos el trinomio cuadrado perfecto:\\

Nos queda:

\begin{center}
     $(x^{2}-2x+1)+(y^{2}+3y+{\frac{9}{4})}=10+1+\frac{9}{4}$
\end{center}

Factorizamos:

\begin{center}
     $(x-1)^{2}+(y+{\frac{3}{2})^{2}}=10+1+\frac{9}{4}$
\end{center}

Resolvemos del lado derecho:

\begin{center}
     $(x-1)^{2}+(y+{\frac{3}{2})^{2}}=\frac{53}{4}$
\end{center}

Hemos llegado a la forma ordinaria de la ecuación de la circunferencia o a la forma con centro fuera del origen:

\begin{center}
     $(x-h)^{2}+(y-k)^{2}=r^{2}$
\end{center}

El centro tiene coordenadas:

\begin{center}
   $C(h,k)$  $\Rightarrow$ $C(1,-\frac{3}{2})$
\end{center}

El radio es:

\begin{center}
   $r^{2}=\frac{53}{4}$  $\Rightarrow$ $r=\frac{\sqrt{53}}{2}$
\end{center}

\newpage
La gráfica de esta ecuación es la siguiente: \\

\textbf{Gráfica:}
\begin{figure}[htb] \centering \includegraphics[scale=.6]{nuclei.png} \end{figure}


\end{document}
